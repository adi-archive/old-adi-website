\documentclass{beamer}
\usepackage{minted}

\hypersetup{colorlinks=true, urlcolor=blue}
\mode<presentation>
{
  \usetheme{Madrid}      % or try Darmstadt, Madrid, Warsaw, ...
  \usecolortheme{beaver} % or try albatross, beaver, crane, ...
  \usefonttheme{default}  % or try serif, structurebold, ...
  \setbeamertemplate{navigation symbols}{}
  \setbeamertemplate{caption}[numbered]
} 

\usepackage[english]{babel}
\usepackage[utf8x]{inputenc}

\title{Intro to \LaTeX}
\author{Kenneth Cheng}
\institute{Columbia University}
\date{March 4, 2013}

\begin{document}

\begin{frame}{Intro to \LaTeX}
  \titlepage
\end{frame}

\begin{frame}[fragile]{What is \LaTeX?}
  \LaTeX\ is a typesetting (think Microsoft Word) tool for:
  \begin{itemize}
    \item mathematical writing
    \item research papers
    \item books
    \item essays
    \item resumes
    \item forms
    \item presentations
  \end{itemize}
  \LaTeX\ takes plain text files and compiles them to PDFs.
\end{frame}

\begin{frame}[fragile]{Why \LaTeX?}
  Why you should use \LaTeX.
  \begin{itemize}
    \item content vs.\ presentation
    \item plain text
    \item research and web presence
    \item looks nice!
  \end{itemize}
  Why you should NOT use \LaTeX.
  \begin{itemize}
    \item steep learning curve
    \item hard to install?
  \end{itemize}
\end{frame}

\begin{frame}[fragile]{Hello World}
  \begin{minted}[gobble=2]{latex}
    \documentclass{article}

    \begin{document}
    Hello world!
    \end{document}
  \end{minted}
\end{frame}

\begin{frame}[fragile]{Preamble}
  \begin{minted}[gobble=2]{latex}
    \documentclass{article}
    \usepackage[margin=1in]{geometry}

    \title{Intro to \LaTeX}
    \author{Kenneth Cheng}
    \date{March 4, 2013}

    \begin{document}
    \maketitle
    Hello world!
    \end{document}
  \end{minted}
\end{frame}

\begin{frame}[fragile]{Content vs.\ Presentation}
  \LaTeX\ uses backslash for all commands and symbols.
  \begin{minted}[gobble=2]{latex}
    \begin{equation}
      \frac{1}{2} \le 1
    \end{equation}
  \end{minted}

  \LaTeX\ treats whitespace similar to HTML.
  \begin{itemize}
    \item
      \begin{minted}[gobble=6]{latex}
        The quick brown fox jumps over the lazy dog.
      \end{minted}
    \item
      \begin{minted}[gobble=6]{latex}
        The    quick brown fox    jumps
        over the lazy dog.
      \end{minted}
  \end{itemize}

  \LaTeX\ supports comments!
  \begin{minted}[gobble=2]{latex}
    This will show
    % This won't show up in the PDF.
    up in the PDF.
  \end{minted}
\end{frame}

\begin{frame}[fragile]{Semantics}
  \LaTeX\ organizes content on the:
  \begin{itemize}
    \item block level in environments.
      \begin{minted}[gobble=6]{latex}
        \begin{itemize}
          \item block level in environments.
          \item inline level in commands.
        \end{itemize}
      \end{minted}
    \item inline level in commands.
      \begin{minted}[gobble=6]{latex}
        \section{Style Guide}
        You can \emph{italicize} and \textbf{bold} fonts.
      \end{minted}
  \end{itemize}
\end{frame}

\begin{frame}[fragile, t]{Math Mode}
  You can write math in math-supported environments.
  \begin{itemize}
    \item Equation
      \begin{minted}[gobble=6]{latex}
        \begin{equation}
          \sum_{i=0}^n i = \frac{n(n+1)}{2}
        \end{equation}
      \end{minted}
      \begin{equation}
        \sum_{i=0}^n i = \frac{n(n+1)}{2}
      \end{equation}
  \end{itemize}
  You can remove the numbering using the \verb|equation*| environment.
  \begin{itemize}
    \item
      \begin{minted}[gobble=6]{latex}
        \begin{equation*}
          \sin \alpha, \Gamma(z) = \int_0^\infty t^{z-1} e^{-t} dt
        \end{equation*}
      \end{minted}
  \end{itemize}
  \begin{equation*}
    \sin \alpha, \Gamma(z) = \int_0^\infty t^{z-1} e^{-t} dt
  \end{equation*}
\end{frame}
\begin{frame}[fragile, t]{Math Mode}
  You can write math in math-supported environments.
  \begin{itemize}
    \item Align
      \begin{minted}[gobble=6]{latex}
        \begin{align}
          \sum_{i=0}^{n+1} i & = (n+1) + \sum_{i=0}^n i \\
          & = (n+1) + \frac{n(n+1)}{2} \\
          & = \frac{(n+2)(n+1)}{2}
        \end{align}
      \end{minted}
      \begin{align}
        \sum_{i=0}^{n+1} i & = (n+1) + \sum_{i=0}^n i \\
        & = (n+1) + \frac{n(n+1)}{2} \\
        & = \frac{(n+2)(n+1)}{2}
      \end{align}
  \end{itemize}
\end{frame}
\begin{frame}[fragile, t]{Math Mode}
  You can write math in math-supported environments.
  \begin{itemize}
    \item Inline \$\$
      \begin{minted}[gobble=6]{latex}
        By induction, $\forall n \in \mathbb{N},
        \sum_{i=0}^n i = \frac{n(n+1)}{2}$.
      \end{minted}
      By induction, $\displaystyle \forall n \in \mathbb{N},
      \sum_{i=0}^n i = \frac{n(n+1)}{2}$.
  \end{itemize}
  Some symbols like $\sum$ render differently inline.
  \begin{itemize}
    \item Add \verb|\displaystyle|
      \begin{minted}[gobble=6]{latex}
        By induction, $\displaystyle \forall n \in \mathbb{N},
        \sum_{i=0}^n i = \frac{n(n+1)}{2}$.
      \end{minted}
    \item or add it to your preamble.
      \begin{minted}[gobble=6]{latex}
        \everymath{\displaystyle}
      \end{minted}
  \end{itemize}
\end{frame}

\begin{frame}[fragile]{Special characters}
  \begin{itemize}
    \item \verb|\$, \#, \%, \&, \~, \_, \^, \textbackslash, \{, \}|
    \item Use \verb|``quote''| for quotations.
  \end{itemize}
\end{frame}

\begin{frame}[fragile]{A few symbols}
  \url{http://www.artofproblemsolving.com/Wiki/index.php/LaTeX:Symbols}
\end{frame}

\begin{frame}[fragile, t]{Tables}
  In math mode (for matrices):
  \begin{itemize}
    \item
      \begin{minted}[gobble=6]{latex}
        \begin{equation*}
          \left(
          \begin{array}{ccc}
            a & b & c \\
            d & e & f \\
            g & h & i
          \end{array}
          \right)
        \end{equation*}
      \end{minted}
      \begin{equation*}
        \left(
        \begin{array}{ccc}
          a & b & c \\
          d & e & f \\
          g & h & i
        \end{array}
        \right)
      \end{equation*}
  \end{itemize}
\end{frame}
\begin{frame}[fragile, t]{Tables}
  NOT in math mode:
  \begin{itemize}
    \item
      \begin{minted}[gobble=6]{latex}
        \begin{tabular}{|l|c|}
          \hline
          operators & $F(t) = \mathcal{L}\{f(t)\}(s)$ \\
          \hline
          vectors & $\mathbf{x} \cdot \hat{\mathbf{n}}$ \\
          \hline
          common sets & $\mathbb{Z}, \mathbb{Q}, \mathbb{R}$ \\
          \hline
        \end{tabular}
      \end{minted}
      \begin{tabular}{|l|c|}
        \hline
        operators & $F(t) = \mathcal{L}\{f(t)\}(s)$ \\
        \hline
        vectors & $\mathbf{x} \cdot \hat{\mathbf{n}}$ \\
        \hline
        common sets & $\mathbb{Z}, \mathbb{Q}, \mathbb{R}$ \\
        \hline
      \end{tabular}
  \end{itemize}
\end{frame}

\begin{frame}[fragile]{TikZ}
  TikZ is your graphics package for
  \begin{itemize}
    \item plots
    \item graphs and trees
    \item diagrams
  \end{itemize}
  \url{http://texample.net/tikz/examples/}
\end{frame}

\begin{frame}[fragile]{Resources}
  \begin{itemize}
    \item \url{http://en.wikibooks.org/wiki/LaTeX}
    \item \url{http://www.artofproblemsolving.com/Wiki/index.php/LaTeX}
    \item \url{http://texample.net/tikz/examples/}
    \item \url{http://tex.stackexchange.com/}
    \item \url{http://www.ctan.org/}
  \end{itemize}
\end{frame}

\end{document}
